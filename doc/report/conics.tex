\chapter{Conics}\label{chap:conics}

\section{Introduction}

Having developed a set of theorems for points and lines, the next most complex geometrical objects to consider are conics.
In this chapter we will present the core theory of conics in planar universal geometry, drawing inspiration from previous work done on conics in chromogeometry\cite{wildberger:rel-conics}.
Throughout this chapter we assume that all values used are taken from a given field $\mathbb{F}$ not of characteristic two.

We begin by establishing some basic definitions and notation.

\begin{definition} Given numbers $A, B, C, D, E, F$, a \emph{conic} is defined as the locus of points $[x,y]$ such that $Ax^2 + Bxy + Cy^2 + Dx + Ey + F = 0$.
\end{definition}

We note that a conic is defined independently of any geometry.
When associated with a particular geometry, a conic may take on particular interesting properties, which will be discussed below.

To assist in working with conics we introduce two notations for their representation.

\begin{definition} The conic which satisfies $Ax^2 + Bxy + Cy^2 + Dx + Ey + F = 0$ is represented by the notations $\langle A\!:\!B\!:\!C\!:\!D\!:\!E\!:\!F \rangle$ or $\langle P\!:\!\vec{q}\!:\!r \rangle$ where
\begin{eqnarray}
P & = & \begin{pmatrix}A & B/2 \\ B/2 & C  \end{pmatrix} \nonumber\\
\vec{q} & = & \begin{pmatrix} D \\ E  \end{pmatrix}\nonumber\\
r & = & F.\nonumber
\end{eqnarray}
\end{definition}
Throughout this chapter we will use the convention that the conic $\langle P\!:\!\vec{q}\!:\!r \rangle$ corresponds to $\langle A\!:\!B\!:\!C\!:\!D\!:\!E\!:\!F \rangle$ without explicitly stating this relation.

\begin{definition} Two conics, $\langle P\!:\!\vec{q}\!:\!r \rangle$ and $\langle P'\!:\!\vec{q}'\!:\!r' \rangle$ are defined to be equivalent if
\begin{eqnarray}
P & = & \lambda P'\\
\vec{q} & = & \lambda\vec{q}'\\
r & = & \lambda r'
\end{eqnarray}
for some $\lambda \neq 0$.
\end{definition}

We also introduce a function $M : \mathbb{F} \times \mathbb{F} \to M_2(\mathbb{F})$ defined by

\begin{eqnarray}
M(a, b) & \mapsto & \begin{pmatrix} a^2 & ab \\ ab & b^2 \end{pmatrix}.
\end{eqnarray}

This function arises naturally in a number of places and has some useful properties.
\begin{itemize}
\item $\det(M(a, b)) = 0$.
\item $M(a, b) = 0$ iff $a = b = 0$.
\item If exactly one of $a$ and $b$ are zero then $M(a, b)$ has one non-zero entry.
\item If both $a$ and $b$ are non-zero then $M(a, b)$ has no zero entries.
\end{itemize}

\section{Tangents, Poles and Polars}

Tangents, while being a geometrical construct, are often considered in the study of analysis, where they are defined in terms of infinitesimal limits.
As has been shown for Euclidean geometry, it is possible to define tangents to conics in a purely algebraic manner\cite{thebook}.
Indeed the definition of a tangent does not depend on a metric, and so the results below serve to express results known from Euclidean geometry in the notation of this thesis.

\begin{definition} A line is a \emph{tangent} to a conic if it intersects that conic at exactly one point.
\end{definition}

\begin{theorem}[Tangent condition] The line $\langle a\!:\!b\!:\!c\rangle$ is tangent to the conic $\langle P\!:\!\vec{q}\!:\!r\rangle$ if and only if
\begin{eqnarray}
(Bc + Db + Ea)^2 -4(ACc^2 + AFb^2 + CFa^2) + & &\nonumber\\
4\left((AE - BD)bc + (CD - BE)ac  + (BF- DE)ab\right) & = & 0.\label{eq:tang}
\end{eqnarray}
\end{theorem}
\begin{proof}
If the line is a tangent then we need to solve the simultaneous equations
\begin{eqnarray*}
Ax^2 + Bxy + Cy^2 + Dx + Ey + F & = & 0\\
ax + by + c & = & 0
\end{eqnarray*}
to find the unique point $X = [x, y]$. From the equation of the line we get $x = -\frac{c + by}{a}$, which, when substituted into the conic equation, gives
\begin{eqnarray*}
0 & = & A\left(\frac{c + by}{a}\right)^2 - B\left(\frac{c + by}{a}\right)y + Cy^2 - D\left(\frac{c + by}{a}\right) + Ey + F\\
  & = & A(c + by)^2 - Ba(c + by)y + Ca^2y^2 - Da(c + by)+ Ea^2y + Fa^2\\
  & = & (Ab^2 - Bab + Ca^2)y^2 + (2Abc - Bac - Dab + Ea^2)y + (Ac^2 - Dac + Fa^2).
\end{eqnarray*}
This equation is quadratic in $y$, so for there to be a unique solution we require the determinant in the quadratic equation to be zero. This leads to the condition
\begin{eqnarray*}
(2Abc - Bac - Dab + Ea^2)^2 & = & 4(Ab^2 - Bab + Ca^2)(Ac^2 - Dac + Fa^2).
\end{eqnarray*}
Algebraic manipulation (omitted here for brevity) leads to the desired result.
\end{proof}

\begin{theorem}[Tangent through a point]\label{th:tangent} Given a point $X_0$ which lies on the conic $\langle P\!:\!\vec{q}\!:\!r \rangle$, the tangent to the conic through the point is $\langle a\!:\!b\!:\!c \rangle$ where
\begin{eqnarray}
\begin{pmatrix}a \\ b\end{pmatrix} & = & \vec{q} + 2P\vec{X_0}\\
c & = & \vec{q}\cdot \vec{X_0} + 2r.
\end{eqnarray}
\end{theorem}
\begin{proof}
The relations above can be shown to satisfy equation (\ref{eq:tang}). The algebraic manipulations required to verify this are best attempted in a computer algebra system since they result in up to 48 terms and as such are omitted here.
\end{proof}

Another metric-free notion which we can borrow from Euclidean geometry is that of poles and polars.

\begin{definition} Given a point $X_0$ and a conic $\langle P\!:\!\vec{q}\!:\!r \rangle$, we can generally construct two tangents to the conic which pass through $X_0$. If we denote the points where these tangents meet the conic as $X_1$ and $X_2$ then the line which passes through $X_1$ and $X_2$ is defined as the \emph{polar} of the point $X_0$, the \emph{pole}.
\end{definition}

\begin{theorem}[Polar from pole]\label{th:polar} Given a point $X_0$ and a conic $\langle P\!:\!\vec{q}\!:\!r \rangle$, the polar of the pole $X_0$, with respect to the conic, is $\langle a\!:\!b\!:\!c \rangle$ where
\begin{eqnarray}
\begin{pmatrix}a \\ b\end{pmatrix} & = & \vec{q} + 2P\vec{X_0}\label{eq:polar:ab}\\
c & = & \vec{q}\cdot\vec{X_0} + 2r.\label{eq:polar:c}
\end{eqnarray}
\end{theorem}
\begin{proof}
From Theorem \ref{th:tangent} we know that the tangent line which passes through through $X_0$ (the pole) and $X_1$, the tangent point on the conic, must satisfy the equation
\begin{eqnarray*}
(\vec{q} + 2P\vec{X_1})\cdot \vec{X_0} + \vec{q}\cdot \vec{X_1} + 2r & = & 0.
\end{eqnarray*}
Likewise, the tangent through $X_2$ and $X_0$ gives
\begin{eqnarray*}
(\vec{q} + 2P\vec{X_2})\cdot \vec{X_0} + \vec{q}\cdot \vec{X_2} + 2r & = & 0.
\end{eqnarray*}
Adding these two equations we get
\begin{eqnarray*}
2\vec{q}\cdot\vec{X_0} + 2(\vec{X_1} + \vec{X_2})P\vec{X_0} + \vec{q}\cdot(\vec{X_1} + \vec{X_2}) + 4r & = & 0\\
(\vec{q} + 2P\vec{X_0})\cdot(\vec{X_1} + \vec{X_2}) + 2(\vec{q}\cdot\vec{X_0} + 2r) & = & 0.
\end{eqnarray*}
As such the line which passes through $X_1$ and $X_2$ is
\begin{eqnarray*}
(\vec{q} + 2P\vec{X_0})\cdot \vec{X} + \vec{q}\cdot \vec{X_0} + 2r & = & 0.
\end{eqnarray*}
\end{proof}

\begin{corollary}
The tangent through a point on a conic is also the polar of that point.
\end{corollary}
\begin{proof}
This follows directly from the equations in Theorems \ref{th:tangent} and \ref{th:polar}.
\end{proof}

\begin{theorem}[Pole from polar]Given a conic $\langle P\!:\!\vec{q}\!:\!r \rangle$ and a polar $\langle a\!:\!b\!:\!c \rangle$, the corresponding pole is
\begin{eqnarray}
\vec{X_0} & = & \frac{1}{2}P^{-1}\begin{pmatrix} \lambda a -D\\ \lambda b-E \end{pmatrix}
\end{eqnarray}
where
\begin{eqnarray}
\lambda & = & \frac{D(CD - BE/2) + E(AE - BD/2) - 4\Delta_PF}{D(Ca - Bb/2) + E(Ab - Aa/2) - 2\Delta_Pc}.
\end{eqnarray}
\end{theorem}
\begin{proof}
Equation (\ref{eq:polar:ab}) gives the vector form of the polar, up to a scale factor. Rearranging this equation we get
\begin{eqnarray*}
\vec{X_0} & = & \frac{P^{-1}}{2}\left(\lambda\begin{pmatrix} a \\ b \end{pmatrix} - \vec{q}\right)\\
    & = & \frac{1}{2\Delta_P}\begin{pmatrix}C & -B/2 \\ -B/2 & A \end{pmatrix}\begin{pmatrix} \lambda a -D\\ \lambda b-E \end{pmatrix}.
\end{eqnarray*}
We can find the scale factor $\lambda$ by combining this result with equation (\ref{eq:polar:c}) to give
\begin{eqnarray*}
\lambda c & = & \vec{q}\cdot \vec{X_0} + 2r\\
 & = & \frac{1}{2\Delta_P}D\left(C(\lambda a - D) - B(\lambda b - E)/2\right) + \\
 &   & \frac{1}{2\Delta_P}E(-B(\lambda a - D)/2 + A(\lambda b -E)) + 2F\\
\lambda & = & \frac{\frac{1}{2\Delta_P}(D(CD - BE/2) + E(AE - BD/2) - 2F}{\frac{1}{2\Delta_P}(D(Ca - Bb/2) + E(Ab - Aa/2)) - c}\\
        & = & \frac{D(CD - BE/2) + E(AE - BD/2) - 4\Delta_PF}{D(Ca - Bb/2) + E(Ab - Aa/2) - 2\Delta_Pc}.
\end{eqnarray*}
\end{proof}


\section{General Conics}

Having established a notation and some metric-free results about conics, we now look at ways of constructing conics based on structures such as points and lines.
These constructions will depend intrinsically on the metric of the geometry.
The results found here will generalise certain definitions and results known from chromogeometry.
We begin with the most general construction.

\begin{definition} Given a point $X_0$ and a line $l_1$ a \emph{general conic} is defined as the locus of points which have a constant ratio of quadrance to $X_0$ and quadrance to $l_1$. 
The ratio, $K$, cannot be zero. 
The point $X_0$ is called the \emph{focus} and the line $l_1$ is the \emph{directrix}.

The points on the conic satisfy the equation
\begin{eqnarray}
Q(X, X_0) & = & KQ(X, l_1).\label{conic_eq}
\end{eqnarray}

\end{definition}

\begin{theorem}[General form of a conic]
The general form of a conic with focus $X_0$, directrix $l_1$ and constant $K$ is $\langle P\!:\!\vec{q}\!:\!r \rangle$ where
\begin{eqnarray}
P & = & G - K\alpha M(a_1, b_1)\label{conic_P}\\
\vec{q} & = & -2\left(G\vec{X_0} + c_1K\alpha\begin{pmatrix}a_1 \\ b_1\end{pmatrix}\right)\label{conic_q}\\
r & = & \vec{X_0}^2 - K\alpha c_1^2\label{conic_r}
\end{eqnarray}
and $\alpha = \Delta_G/\|\vec{l_1}\|$.

\end{theorem}
\begin{proof}
Expanding equation (\ref{conic_eq}) we get
\begin{eqnarray}
Q(X, X_0) & = & KQ(X, l_1)\nonumber\\
\vec{X}\cdot\vec{X} -2\vec{X}\cdot\vec{X_0} + \vec{X_0}^2 & = & K\frac{l_1(X)^2}{\|\vec{l_1}\|}\Delta_G\nonumber\\
 & = & K\alpha l_1(X)^2\nonumber\\
 & = & K\alpha(a_1x + b_1y + c_1)^2\nonumber\\
 & = & K\alpha((a_1^2x^2  + 2a_1b_1xy + b_1^2y^2) + 2c_1(a_1x + b_1y) + c_1^2).\nonumber
\end{eqnarray}
\end{proof}

The quadratic term $P$ in any conic can tell us a lot about the nature of the conic, as we see here and below when we consider circles and parabolas.

\begin{lemma}\label{lemma:lambda1}
A general conic $\langle P\!:\!\vec{q}\!:\!r \rangle$ satisfies $P \neq G$.
\end{lemma}
\begin{proof}
If $P = G$ then from (\ref{conic_P}) we require $K\alpha M(a_1, b_1) = 0$ for some $l_1 = \langle a_1\!:\!b_1\!:\!c_1 \rangle$.
From the definition of a line we must have either $a_1 \neq 0$ or $b_1 \neq 0$ so we require either $K = 0$ or $\alpha = 0$.
We know that $\alpha \neq 0$ since $\Delta_G \neq 0$. 
We also know that $K \neq 0$ by definition and therefore $P \neq G$.
\end{proof}

\begin{lemma}
A general conic $\langle P\!:\!\vec{q}\!:\!r \rangle$ satisfies $G \neq \lambda P$ for $\lambda \neq 0$.
\end{lemma}
\begin{proof}
Assume that $G = \lambda P$. From Lemma \ref{lemma:lambda1} we can take $\lambda \neq 1$. From equation (\ref{conic_P}) we have
\begin{eqnarray}
P & = & G - K\alpha M(a_1, b_1)\nonumber\\
\left(1 - \frac{1}{\lambda} \right)G & = & K\alpha M(a_1, b_1).\nonumber
\end{eqnarray}
Taking the determinant of each side we get
\begin{eqnarray*}
\left(1 - \frac{1}{\lambda} \right)\Delta_G & = & K\alpha \det(M(a_1, b_1))\\
 & = & 0
\end{eqnarray*}
which is a contradiction, since by definition $\Delta_G \neq 0$.
\end{proof}

Some theorems below will require that the matrix $G - P$, for some conic $\langle P\!:\!\vec{q}\!:\!r \rangle$ in a geometry $G$, has no zero entries.
Pre-empting this, we present the following two results.

\begin{lemma}
For a general conic $\langle P\!:\!\vec{q}\!:\!r \rangle$ the matrix $G - P$ has either 1 or 4 non-zero entries.
\end{lemma}
\begin{proof}
Since $G - P = K\alpha M(a_1, b_1)$ and at least one of $a_1$, $b_1$ are non-zero the result follows directly from the properties of the function $M$.
\end{proof}

\begin{lemma}
For a general conic $\langle P\!:\!\vec{q}\!:\!r \rangle$, if the matrix $G - P$ has 1 non-zero entry, there exists an equivalent conic $\langle P'\!:\!\vec{q}'\!:\!r' \rangle$ such that $G - P'$ has 4 non-zero entries.
\end{lemma}
\begin{proof}
If we pick $\lambda \notin \{0, c/C, a/A\}$ and let $P' = \lambda P$ then
\begin{eqnarray*}
G - P' & = & \left(\begin{array}{cc} a - \lambda A & b - \lambda B/2 \\ b - \lambda B/2 & c - \lambda C \end{array}\right).
\end{eqnarray*}
From our choice of $\lambda$ we know that $a - \lambda A \neq 0$ and $c - \lambda C \neq 0$, which means $G - P'$ has at least two non-zero entries. Since $G-P'$ must have either one or four non-zero entries, it must have four non-zero entries.
\end{proof}

Having found a way to construct conics from a focus and a directrix, we would like to be able to, given a conic, recover the focus and directrix used to construct it.
The remainder of this section addresses this problem.

\begin{definition}If $l_1$ is a directrix of a conic then $\vec{l_1}$ is a \emph{directrix vector} of the conic.
\end{definition}

\begin{theorem}[Direction of directrices]\label{th:dirdir}
The conic $\langle P\!:\!\vec{q}\!:\!r \rangle$ has directrix vectors $\vec{l_1} = \begin{pmatrix}-b_1\\a_1\end{pmatrix}$ and $\vec{l_2} = \begin{pmatrix}-b_2\\a_2\end{pmatrix}$ where
\begin{eqnarray}
\left(\begin{array}{cc}a_1 & a_2\\b_1 & b_2 \end{array} \right) & = & G - P\\
 & = & K\alpha M(a_1, b_1).
\end{eqnarray}
\end{theorem}
\begin{proof}
We wish to solve equation (\ref{conic_P}) for $a_1$ and $b_1$.
Without loss of generality we can assume that $G - P$ has no zero entries, which means that $a_1$ are $b_1$ are both non-zero. This lets us find an expression for $K\alpha$ by considering the off-diagonal elements of our matrix equation.
\begin{eqnarray}
P & = & G - K\alpha M(a_1, b_1)\nonumber\\
\left(\begin{array}{cc} a - A & b - B/2 \\ b - B/2 & c - C \end{array} \right) & = &  K\alpha\left(\begin{array}{cc}a_1^2 & a_1b_1 \\ a_1b_1 & b_1^2 \end{array} \right)\nonumber\\
K\alpha & = & \frac{b - B/2}{a_1b_1}.\label{eq:K}
\end{eqnarray}
We can now use this value to equate the diagonal elements of the matrices above, giving
\begin{eqnarray*}
a_1(b - B/2) & = & b_1(a - A)\\
b_1(b - B/2) & = & a_1(c - C).
\end{eqnarray*}
There are two possible solutions to these equations. $a_1 = a - A$, $b_1 = b - B/2$ or $a_1 = b - B/2$, $b_1 = c - C$.
Putting these results into matrix form gives the desired result.
\end{proof}

\begin{lemma}
The two directrix vectors of a conic are parallel.
\end{lemma}
\begin{proof}
Since $G - P = K\alpha M(a_1, b_1)$ and $\det(M(a_1, b_1)) = 0$, we must have $\left|\begin{array}{cc} a_1 & a_2 \\ b_1 & b_2 \end{array}\right| = a_1b_2 - a_2b_1 = 0$ and therefore the directrix vectors are parallel.
\end{proof}

Although we have not yet found the focus and directrix, we are now in a position to determine the constant $K$.

\begin{theorem}[Constant of focus/directrix pairs] 
The constant $K$ associated with the focus/directrix pairs of $\langle P\!:\!\vec{q}\!:\!r \rangle$ is
\begin{eqnarray}
K & = & \frac{a(b - B/2)^2 -2b(a - A)(b - B/2) + c(a - A)^2}{(a - A)\Delta_G}.
\end{eqnarray}

\end{theorem}
\begin{proof}From Theorem \ref{th:dirdir} and equation (\ref{eq:K}) we get
\begin{eqnarray}
K & = & \frac{b - B/2}{\alpha a_1b_1}\nonumber\\
  & = & \frac{b - B/2}{\alpha(a - A)(b - B/2)}\nonumber\\
  & = & \frac{\|\vec{l_1}\|}{\alpha(a - A)\Delta_G}\nonumber\\
  & = & \frac{a(b - B/2)^2 -2b(a - A)(b - B/2) + c(a - A)^2}{(a - A)\Delta_G}.\nonumber
\end{eqnarray}
\end{proof}

We have shown that the two directrix vectors are parallel, so in actual fact we only have a single direction for our directrices.
Although in Euclidean geometry we expect a conic to have two focus and directrix pairs, to prove this in planar universal geometry we need the following result.
\begin{theorem}
Given a conic $\langle P\!:\!\vec{q}\!:\!r \rangle$ with directrix vector $\vec{l_1}$ and constant $K$, its two directrices are $l_1 = \langle a_1\!:\!b_1\!:\!c_{11} \rangle$ and $l_2 = \langle a_1\!:\!b_1\!:\!c_{12} \rangle$ where $c_{11}$ and $c_{12}$ are the roots of the equation
\begin{eqnarray}
0 & = & 4K\alpha(\|\vec{l_1}\|K\alpha - \Delta_G )c_1^2 + 4K\alpha(\vec{Q}\cdot\vec{l_1})c_1 + (\vec{Q}\cdot\vec{Q} - 4\Delta_GF)\label{eq:direc_c}
\end{eqnarray}
where
\begin{eqnarray}
\vec{Q} & = & \begin{pmatrix} -E \\ D \end{pmatrix}\\
\alpha & = & \frac{\Delta_G}{\|\vec{l_1}\|}.
\end{eqnarray}
\end{theorem}
\begin{proof}
We need to solve equations (\ref{conic_q}) and (\ref{conic_r}) for $c_1$. We begin by finding an expression for $X_0$.
\begin{eqnarray}
\vec{q} & = & -2\left(G\vec{X_0} + c_1K\alpha\begin{pmatrix}a_1 \\ b_1\end{pmatrix}\right)\nonumber\\
2G\vec{X_0} & = & -\vec{q} -2c_1K\alpha\begin{pmatrix}a_1 \\ b_1\end{pmatrix}\nonumber\\
2\vec{X_0} & = & -G^{-1}\left(\vec{q} + 2c_1K\alpha\begin{pmatrix}a_1 \\ b_1\end{pmatrix} \right).\label{eq:focus}
\end{eqnarray}
We need to find $\vec{X_0}^2$ so we take the metric dot product of each side of the above equation with itself and obtain
\begin{eqnarray*}
(2\vec{X_0})^T(2G\vec{X_0}) & = & \left(G^{-1}\left(\vec{q} + 2c_1K\alpha\begin{pmatrix} a_1 \\ b_1 \end{pmatrix} \right)\right)^T\left(\vec{q} + 2c_1K\alpha\begin{pmatrix} a_1 \\ b_1 \end{pmatrix} \right)\nonumber\\
4\vec{X_0}^2 & = & \frac{1}{\Delta_G} \begin{pmatrix} c(D + 2a_1c_1K\alpha) - b(E + 2b_1c_1K\alpha) \\ -b(D + 2a_1c_1K\alpha) + a(E + 2b_1c_1K\alpha)  \end{pmatrix} \begin{pmatrix} D + 2a_1c_1K\alpha \\ E + 2b_1c_1K\alpha \end{pmatrix}\\
4\Delta_G\vec{X_0}^2 & = & c(D + 2a_1c_1K\alpha)^2 -2b(D + 2a_1c_1K\alpha)(E + 2b_1c_1K\alpha) + \\
 &  & a(E + 2b_1c_1K\alpha)^2.
\end{eqnarray*}
We are now able to solve equation (\ref{conic_r}) in terms of $c_1$.
\begin{eqnarray*}
r & = & \vec{X_0}^2 - K\alpha c_1^2\\
4\Delta_G(F + c_1^2K\alpha) & = & c(D + 2a_1c_1K\alpha)^2 -2b(D + 2a_1c_1K\alpha)(E + 2b_1c_1K\alpha) + \\
  &  & a(E + 2b_1c_1K\alpha)^2\\
0 & = & (4ca_1^2K^2\alpha^2 - 8ba_1b_1K^2\alpha^2 + 4ab_1K^2\alpha^2 - 4\Delta_GK\alpha )c_1^2 \\
  &   & + (4cDa_1K\alpha -2b(2Db_1K\alpha + 2Ea_1K\alpha) + 4aEb_1K\alpha)c_1 \\
  &   & + (cD^2 -2bDE + aE^2 - 4\Delta_GF)\\
  & = & 4K\alpha(\|\vec{l_1}\|K\alpha - \Delta_G )c_1^2 + 4K\alpha(\vec{Q}\cdot\vec{l_1})c_1 + (\vec{Q}\cdot\vec{Q} - 4\Delta_GF).
\end{eqnarray*}
\end{proof}

Having found that there are indeed two directrices, finding their associated foci is relatively simple.

\begin{theorem}[Focus of a directrix]\label{th:focus}
Given a conic $\langle P\!:\!\vec{q}\!:\!r \rangle$ with directrix $l_1 = \langle a_1\!:\!b_1\!:\!c_1 \rangle$ and constant $K$, the associated focus is
\begin{eqnarray}
\vec{X_0} & = & -\frac{1}{2}G^{-1}\left(\vec{q} + 2c_1K\alpha\begin{pmatrix}a_1 \\ b_1\end{pmatrix} \right).
\end{eqnarray}
\end{theorem}
\begin{proof}
This result follows directly from (\ref{eq:focus}).
\end{proof}


\section{Circles}

In the construction of conics from focus and directrix we found that we were unable to construct conics such that $G = \lambda P$. We now investigate another way of constructing conics which complements the focus/directrix construction.

\begin{definition} Given a geometry, $G$, a \emph{circle} is defined as the locus of points which are quadrance $K$ from a fixed point $X_0$, i.e. those points which satisfy
\begin{eqnarray}
Q(X, X_0) & = & K.\label{circ_eq}
\end{eqnarray}
The point $X_0$ is the \emph{centre} of the circle and $K$ is its \emph{quadrance}.

\end{definition}

\begin{theorem}[General form of a circle]
The general form of a circle with radius $K$ and centre $X_0$ is $\langle P\!:\!\vec{q}\!:\!r \rangle$ where
\begin{eqnarray}
P & = & G\\
\vec{q} & = & -2G\vec{X_0}\label{circ_q}\\
r & = & \vec{X_0}^2 - K.\label{circ_r}
\end{eqnarray}
\end{theorem}
\begin{proof}
Expanding (\ref{circ_eq}) we get
\begin{eqnarray}
Q(X, X_0) & = & K\nonumber\\
(\vec{X} - \vec{X_0})\cdot(\vec{X} - \vec{X_0}) & = & K\nonumber\\
\vec{X}\cdot\vec{X} - 2\vec{X}\cdot\vec{X_0} + \vec{X_0}^2 -K & = & 0.\nonumber
\end{eqnarray}
\end{proof}

\begin{lemma}\label{lem:circ}
A conic $\langle P\!:\!\vec{q}\!:\!r \rangle$ is not a circle in the geometry $G$ if $G \neq \lambda P$ for some $\lambda \neq 0$.
\end{lemma}
\begin{proof}
If $\langle P\!:\!\vec{q}\!:\!r \rangle$ is a circle, then from the general form given above there exists an equivalent conic $\langle P'\!:\!\vec{q}'\!:\!r' \rangle$ such that $P' = \lambda P = G$ for $\lambda \neq 0$. If no such $P'$ can be found then the conic is not a circle.
\end{proof}

\begin{theorem}\label{th:circ}
If a conic $\langle P\!:\!\vec{q}\!:\!r \rangle$ in the geometry $G$ satisfies $\lambda P = G$ with $\lambda \neq 0$ then it is a circle in $G$ with centre and quadrance given by

\begin{eqnarray}
\vec{X_0} & = & -\frac{\lambda}{2}G^{-1}\vec{q}\\
K & = & \vec{X_0}^2 - \lambda r.
\end{eqnarray}

\end{theorem}
\begin{proof}
These results follow directly from (\ref{circ_q}) and (\ref{circ_r}).
\end{proof}

\begin{corollary}
A conic $\langle P\!:\!\vec{q}\!:\!r \rangle$ is a circle in a geometry $G$ if and only if $G = \lambda P$ with $\lambda \neq 0$.
\end{corollary}
\begin{proof}
This is essentially a restatement of Lemma \ref{lem:circ} and Theorem \ref{th:circ} combined.
\end{proof}

\begin{corollary}
Every conic $\langle P\!:\!\vec{q}\!:\!r \rangle$ where $\det(P) \neq 0$ is a circle in some geometry G.
\end{corollary}
\begin{proof}
If we let $G = P$ then $G$ is a valid geometry, since $\det(G) \neq 0$ and $P$ is a circle in $G$.
\end{proof}

A result of Euclidean geometry which is taught to all high school students is that the radius of a circle to a point on the circle is perpendicular to the tangent at that point.
In planar universal geometry we can obtain a more general result, which contains the equivalent of the radius-tangent rule as a special case.

\begin{theorem}[Pole to a circle] Given a point $X_1$ and a circle with centre $X_0$, the polar of $X_1$ is $l_1$ where
\begin{eqnarray}
\begin{pmatrix}a_1 \\ b_1\end{pmatrix} & = & 2G(\vec{X_1} - \vec{X_0})\\
c_1 & = & 2\vec{X_0}G(\vec{X_0} - \vec{X_1}) - 2K.
\end{eqnarray}
\end{theorem}
\begin{proof} Combining the equations for the polar from Theorem \ref{th:polar} and the general form of a circle we get
\begin{eqnarray*}
\begin{pmatrix}a_1 \\ b_1\end{pmatrix} & = & \vec{q} + 2P\vec{X_1}\\
 & = & -2G\vec{X_0} + 2G\vec{X_1}\\
 & = & 2G(\vec{X_1} - \vec{X_0})\\
c_1 & = & \vec{q}\cdot\vec{X_1} + 2r\\
    & = & (-2G\vec{X_0})\cdot\vec{X_1} + 2(\vec{X_0}^2 - K)\\
    & = & 2\vec{X_0}G(\vec{X_0} - \vec{X_1}) - 2K.
\end{eqnarray*}
\end{proof}

\begin{theorem}\label{th:perp-polar}Given a point $X_1$ and a circle with centre $X_0$, the line which passes through the centre of the circle and $X_1$ is perpendicular to the polar of $X_1$.
\end{theorem}
\begin{proof}We let $r = \langle y_0 - y_1\!:\!x_1 - x_0 \!:\! x_0y_1 - x_1y_0 \rangle$ be the line through the centre, $X_0$ and the point $X_1$. If we denote the tangent vector as $\vec{t}$ then from the previous theorem we have
\begin{eqnarray*}
\vec{t}         & = & 2\begin{pmatrix}-b(x_0 - x_1) - c(y_0 - y_1) \\ a(x_0 - x_1) + b(y_0 - y_1)\end{pmatrix}\\
\vec{r} & = & \begin{pmatrix}x_0 - x_1 \\ y_0 - y_1\end{pmatrix}\\
\vec{t}\cdot\vec{r} & = & 2\begin{pmatrix}-b(x_0 - x_1) - c(y_0 - y_1) \\ a(x_0 - x_1) + b(y_0 - y_1)\end{pmatrix}  \begin{pmatrix}a & b \\ b & c\end{pmatrix} \begin{pmatrix}x_0 - x_1 \\ y_0 - y_1\end{pmatrix}\\
                & = & 2\begin{pmatrix}-b(x_0 - x_1) - c(y_0 - y_1) \\ a(x_0 - x_1) + b(y_0 - y_1)\end{pmatrix}  \begin{pmatrix}a(x_0 - x_1) + b(y_0 - y_1) \\ b(x_0 - x_1) + c(y_0 - y_1)\end{pmatrix}\\
                & = & 0.
\end{eqnarray*}
\end{proof}

\begin{corollary} Given a point $X_1$ on a circle, the tangent passing through this point is perpendicular to the line through $X_1$ and the centre of the circle.
\end{corollary}
\begin{proof}
Since the tangent through $X_1$ is also the polar of $X_1$ this follows directly from Theorem \ref{th:perp-polar}.
\end{proof}

\section{Parabola}

Probably the first quadratic equation a maths student will meet is $y = x^2$, which describes a parabola in Euclidean geometry.
As one might expect, we can construct such objects in planar universal geometry.

\begin{definition}A \emph{parabola} is the locus of points whose quadrance to a point $X_0$ is equal to its quadrance to a line $l_1$. A parabola is essentially a general conic with $K = 1$.
\end{definition}
\begin{theorem}
The general form of a parabola with focus $X_0$, directrix $l_1$ is $\langle P\!:\!\vec{q}\!:\!r \rangle$ where
\begin{eqnarray}
P & = & G - \alpha M(a_1, b_1)\label{eq:parab_P}\\
\vec{q} & = & -2(GX_0 + c_1\alpha(a_1, b_2))\\
r & = & X_0^2 - \alpha c_1^2.
\end{eqnarray}
\end{theorem}

\begin{proof}
This result follows directly from letting $K=1$ in equations (\ref{conic_P})-(\ref{conic_r}).
\end{proof}

\begin{theorem}\label{th:para}
A conic $\langle P\!:\!\vec{q}\!:\!r \rangle$ is a parabola if and only if $\det(P) = 0$.
\end{theorem}
\begin{proof}
Taking determinants of each side in equation (\ref{eq:parab_P}) yields
\begin{eqnarray}
\left|P\right| & = & \left|G - \alpha M(a_1, b_1)\right|\nonumber\\
 & = & (a - \alpha a_1^2)(c - \alpha b_1^2) - (b - \alpha a_1b_1)^2\nonumber\\
 & = & ac - b^2 - \alpha(ca_1^2 - 2ba_1b_1 + ab_1^2) + \alpha^2(a_1^2b_1^2 - a_1^2b_1^2)\nonumber\\
 & = & \Delta_G - \frac{\Delta_G}{\|\vec{l_1}\|}\|\vec{l_1}\|\nonumber\\
 & = & 0.\nonumber
\end{eqnarray}
\end{proof}

\begin{corollary}
If a conic is a parabola in any geometry then it is a parabola in all geometries.
\end{corollary}
\begin{proof}
This result follows direction from the fact that the conditions in Theorem \ref{th:para} are independent of the geometry.
\end{proof}

\begin{theorem}\label{th:single}A parabola has only a single directrix/focus pair.
\end{theorem}
\begin{proof}
To find the directrices of a conic we need to solve equation (\ref{eq:direc_c}), which is
\begin{eqnarray*}
0 & = & 4K\alpha(\|\vec{l_1}\|K\alpha - \Delta_G )c_1^2 + 4K\alpha(\vec{Q}\cdot\vec{l_1})c_1 + (\vec{Q}\cdot\vec{Q} - 4\Delta_GF).
\end{eqnarray*}
For a parabola with $K=1$ we have $\|\vec{l_1}\|K\alpha - \Delta_G = \|\vec{l_1}\|\frac{\Delta_G}{\|\vec{l_1}\|} - \Delta_G = 0$, which reduces the quadratic equation to a linear equation, which will only have one root.
\end{proof}

\begin{theorem} The focus and directrix of a parabola $\langle P\!:\!\vec{q}\!:\!r\rangle$ are $l_1$ and $X_0$ where 
\begin{eqnarray}
a_1 & = & a - A\\
b_1 & = & b - B/2\\
c_1 & = & \frac{4\Delta_GF - \vec{Q}\cdot\vec{Q}}{4\alpha(\vec{Q}\cdot\vec{l_1})}\\
\vec{X_0} & = &  -\frac{1}{2}G^{-1}\left(\vec{q} + 2c_1\alpha\begin{pmatrix}a_1 \\ b_1\end{pmatrix} \right)
\end{eqnarray}
and
\begin{eqnarray}
\vec{Q} & = & \begin{pmatrix} -E \\ D \end{pmatrix}\\
\alpha & = & \frac{\Delta_G}{\|\vec{l_1}\|}.
\end{eqnarray}
\end{theorem}
\begin{proof}
The results for $a_1$, $b_1$ and $X_0$ come directly from Theorems \ref{th:dirdir} and \ref{th:focus} for a general conic.
From the proof of Theorem \ref{th:single} we have seen that $c_1$ is the solution of the equation
\begin{eqnarray*}
0 & = & 4K\alpha(\vec{Q}\cdot\vec{l_1})c_1 + (\vec{Q}\cdot\vec{Q} - 4\Delta_GF)
\end{eqnarray*}
from which the result directly follows.
\end{proof}

\section{Grammolas}

Conic sections in Euclidean geometry are generally described as parabolas, hyperbolas and ellipses.
While planar universal geometry has parabolas as a fundamental object, the notion of hyperbolas and ellipses do not manifest themselves as distinct objects.
The simplest example to demonstrate this is the unit circle in the green geometry of chromogeometry.
This object has the equation $xy - 1 = 0$, and is, by definition, a circle in the green geometry.
The same equation considered in the blue geometry (i.e. regular Euclidean geometry) is not a circle, but a hyperbola.

In place of hyperbolas and ellipses, we have objects known as grammolas and quadrolas.
These objects have been studied in the context of chromogeometry\cite{wildberger:rel-conics}, and are presented here in the context of planar universal geometry.

\begin{definition} A \emph{grammola} is defined as the locus of points such that the sum of quadrances from each point to two given lines is a constant. Given two lines $d_1$ and $d_2$, the \emph{diagonals}, and a constant $K$, the points on a grammola satisfy the equation
\begin{eqnarray}
Q(X, d_1) + Q(X, d_2) = K.
\end{eqnarray}
\end{definition}

\begin{theorem} Given non-null diagonals $d_1 = \langle a_1\!:\!b_1\!:\!c_1 \rangle$, $d_2 = \langle a_2\!:\!b_2\!:\!c_2 \rangle$ and constant $K$, the general form of the grammola satisfying $Q(X, d_1) + Q(X, d_2) = K$ is $\langle P\!:\!\vec{q}\!:\!r \rangle$ where
\begin{eqnarray}
P & = & \|\vec{d_1}\|M(a_2, b_2) + \|\vec{d_2}\|M(a_1, b_1)\label{eq:gram-P}\\
\vec{q} & = & 2\begin{pmatrix} a_1 & a_2 \\ b_1 & b_2 \end{pmatrix}\begin{pmatrix} \|\vec{d_2}\|c_1 \\ \|\vec{d_1}\|c_2  \end{pmatrix}\label{eq:gram-q}\\
r & = & c_1^2\|\vec{d_2}\| + c_2^2\|\vec{d_1}\| - \frac{K\|\vec{d_1}\|\|\vec{d_2}\|}{\Delta_G}\label{eq:gram-r}
\end{eqnarray}
\end{theorem}
\begin{proof} Starting from the definition and using equation (\ref{eq:point-line}) we find
\begin{eqnarray*}
0 & = & Q(X, d_1) + Q(X, d_2) - K\\
  & = & \frac{d_1(X)^2}{\|\vec{d_1}\|}\Delta_G + \frac{d_2(X)^2}{\|\vec{d_2}\|}\Delta_G - K\\
  & = & \|\vec{d_2}\|(a_1x + b_1y + c_1)^2 + \|\vec{d_1}\|(a_2x + b_2y + c_2)^2 - \frac{K\|\vec{d_1}\|\|\vec{d_2}\|}{\Delta_G}.
\end{eqnarray*}
From this expression the result follows immediately.
\end{proof}

As with the general conics before, we would like to be able to take a grammola and find its diagonals.

\begin{definition}
A vector $\vec{d_1} = \begin{pmatrix} -b_1 \\ a_1 \end{pmatrix}$ is a \emph{diagonal vector} of the conic $\langle P\!:\!\vec{q}\!:\!r \rangle$ if and only if
\begin{eqnarray}
\frac{2A\|\vec{d_1}\| - a_1^2(Ac - Bb + Ca)}{2\|\vec{d_1}\|^2} & = & \lambda^2\\
\frac{2C\|\vec{d_1}\| + b_1^2(Ac - Bb + Ca)}{2\|\vec{d_1}\|^2} & = & \mu^2
\end{eqnarray}
for some $\lambda, \mu \in \mathbb{F}$. The motivation for this definition will become clear below.
\end{definition}
\begin{definition}
If $d_1$ and $d_2$ are the diagonals of a conic then $d_1$ is the \emph{co-diagonal} of $d_2$ and vice-versa. Likewise, $\vec{d_1}$ is the \emph{co-diagonal vector} of $\vec{d_2}$ and vice-versa.
\end{definition}

\begin{theorem}\label{th:gram-covec}
If $\vec{d_1}$ is a diagonal vector of the conic $\langle P\!:\!\vec{q}\!:\!r \rangle$ then its co-diagonal vector is $\vec{d_2}$ where
\begin{eqnarray}
a_2^2 & = & \frac{2A\|\vec{d_1}\| - a_1^2(Ac - Bb + Ca)}{2\|\vec{d_1}\|^2}\\
b_2^2 & = & \frac{2C\|\vec{d_1}\| + b_1^2(Ac - Bb + Ca)}{2\|\vec{d_1}\|^2}\\
\frac{a_2}{b_2} & = & \frac{2A\|\vec{d_1}\| - a_1^2(Ac - Bb + Ca)}{B\|\vec{d_1}\| - a_1b_1(Ac - Bb + Ca)}.
\end{eqnarray}
\end{theorem}
\begin{proof}
To complete this proof we need to solve the system of equations represented in equation (\ref{eq:gram-P}). We start by finding expressions for $a_2^2$, $b_2^2$ and $a_2b_2$.
\begin{eqnarray*}
A & = & \|\vec{d_1}\|a_2^2 + \|\vec{d_2}\|a_1^2\\
  & = & \|\vec{d_1}\|a_2^2 + (ab_2^2 - 2ba_2b_2 + ca_2^2)a_1^2\\
B & = & 2(\|\vec{d_1}\|a_2b_2 + \|\vec{d_2}\|a_1b_1)\\
  & = & 2\|\vec{d_1}\|a_2b_2 + 2(ab_2^2 - 2ba_2b_2 + ca_2^2)a_1b_1\\
C & = & \|\vec{d_1}\|b_2^2 + \|\vec{d_2}\|b_1^2\\
  & = & \|\vec{d_1}\|b_2^2 + (ab_2^2 - 2ba_2b_2 + ca_2^2)b_1^2\\
a_2^2 & = & \frac{A - aa_1^2b_2^2 + 2ba_1^2a_2b_2}{\|\vec{d_1}\| + ca_1^2}\\
a_2b_2 & = & \frac{B - 2aa_1b_1b_2^2 - 2ca_1a_2^2b_1}{2\|\vec{d_1}\| - 4ba_1b_1}\\
b_2^2 & = & \frac{C - ca_2^2b_1^2 + 2ba_2b_1^2b_2}{\|\vec{d_1}\| + ab_1^2}.
\end{eqnarray*}
At this stage we introduce a new symbol, $w = \|\vec{d_1}\| + ab_1^2$ to ease the algebra. Eliminating $b_2^2$ we get
\begin{eqnarray*}
a_2^2 & = & \frac{A - aa_1^2\left(\frac{C - ca_2^2b_1^2 + 2ba_2b_1^2b_2}{\|\vec{d_1}\| + ab_1^2} \right) + 2ba_1^2a_2b_2}{\|\vec{d_1}\| + ca_1^2}\\
(\|\vec{d_1}\| + ab_1^2)(\|\vec{d_1}\| + ca_1^2)a_2^2 & = & Aw - aa_1^2(C - ca_2^2b_1^2 + 2ba_2b_1^2b_2) + \\
 &  & 2ba_1^2a_2b_2w\\
((\|\vec{d_1}\| + ab_1^2)(\|\vec{d_1}\| + ca_1^2) - aca_1^2b_1^2)a_2^2 & = & Aw - Caa_1^2 + 2ba_1^2a_2b_2(w - ab_1^2)\\
\|\vec{d_1}\|(w + ca_1^2)a_2^2 & = & Aw - Caa_1^2 + 2ba_1^2a_2b_2\|\vec{d_1}\|\\
\end{eqnarray*}
and
\begin{eqnarray*}
a_2b_2 & = & \frac{B - 2aa_1b_1\left(\frac{C - ca_2^2b_1^2 + 2ba_2b_1^2b_2}{\|\vec{d_1}\| + ab_1^2}\right) - 2ca_1a_2^2b_1}{2\|\vec{d_1}\| - 4ba_1b_1}\\
(\|\vec{d_1}\| + ab_1^2)(2\|\vec{d_1}\| - 4ba_1b_1)a_2b_2 & = & Bw - 2aa_1b_1(C - ca_2^2b_1^2 + 2ba_2b_1^2b_2) - \\
 &  & 2ca_1a_2^2b_1w\\
((\|\vec{d_1}\| + ab_1^2)(2\|\vec{d_1}\| - 4ba_1b_1) + &  & \\
4aba_1b_1^3 ) a_2b_2 & = & Bw - 2Caa_1b_1 - 2ca_1a_2^2b_1(w - ab_1^2)\\
2\|\vec{d_1}\|(w - 2ba_1b_1) a_2b_2 & = & Bw - 2Caa_1b_1 - 2ca_1a_2^2b_1\|\vec{d_1}\|\\
a_2b_2 & = & \frac{Bw - 2Caa_1b_1 - 2ca_1a_2^2b_1\|\vec{d_1}\|}{2\|\vec{d_1}\|(w - 2ba_1b_1)}.
\end{eqnarray*}
We introduce a new variable $z = Ac - Bb + Ca$ to further simplify the algebra. Substituting the previous expression into our equation for $a_2^2$ we get
\begin{eqnarray*}
\|\vec{d_1}\|(w + ca_1^2)a_2^2 & = & (Aw - Caa_1^2) + ba_1^2\left(\frac{Bw - 2Caa_1b_1 - 2ca_1a_2^2b_1\|\vec{d_1}\|}{(w - 2ba_1b_1)} \right)\\
\|\vec{d_1}\|(w - 2ba_1b_1)(w + ca_1^2)a_2^2 & = & (w - 2ba_1b_1)(Aw - Caa_1^2) + ba_1^2(Bw - 2Caa_1b_1)\\
 &  &  - 2bca_1^3a_2^2b_1\|\vec{d_1}\|\\
\|\vec{d_1}\|((w - 2ba_1b_1)(w + ca_1^2)  + &  & \\
2bca_1^3b_1 )a_2^2 & = & Aw^2 - 2Aba_1b_1w - Caa_1^2w + ba_1^2Bw\\
\|\vec{d_1}\|w(w - 2ba_1b_1 + ca_1^2)a_2^2 & = & Aw^2 - 2Aba_1b_1w - Caa_1^2w + ba_1^2Bw\\
2\|\vec{d_1}\|^2a_2^2 & = & A(w - 2ba_1b_1) - Caa_1^2 + ba_1^2B\\
2\|\vec{d_1}\|^2a_2^2 & = & 2A\|\vec{d_1}\| - Aca_1^2 - Caa_1^2 + ba_1^2B\\
2\|\vec{d_1}\|^2a_2^2 & = & 2A\|\vec{d_1}\| - a_1^2z \\
a_2^2 & = & \frac{2A\|\vec{d_1}\| - a_1^2z}{2\|\vec{d_1}\|^2}.
\end{eqnarray*}
We can now use this expression to find $a_2b_2$.
\begin{eqnarray*}
a_2b_2 & = & \frac{Bw - 2Caa_1b_1 - 2ca_1b_1\left(\frac{2A\|\vec{d_1}\| - a_1^2z}{2\|\vec{d_1}\|^2} \right)\|\vec{d_1}\|}{2\|\vec{d_1}\|(w - 2ba_1b_1)}\\
       & = & \frac{\|\vec{d_1}\|(Bw - 2Caa_1b_1) - ca_1b_1(2A\|\vec{d_1}\| - a_1^2z)}{2\|\vec{d_1}\|^2(2\|\vec{d_1}\| - ca_1^2)}\\
       & = & \frac{\|\vec{d_1}\|(B(2ab_1^2 - 2ba_1b_1 + ca_1^2) - 2Caa_1b_1 - 2Aca_1b_1) + ca_1^3b_1z}{2\|\vec{d_1}\|^2(2\|\vec{d_1}\| - ca_1^2)}\\
       & = & \frac{\|\vec{d_1}\|(B(2ab_1^2 - 4ba_1b_1 + ca_1^2) - 2a_1b_1z) + ca_1^3b_1z}{2\|\vec{d_1}\|^2(2\|\vec{d_1}\| - ca_1^2)}\\
       & = & \frac{\|\vec{d_1}\|B(2ab_1^2 - 4ba_1b_1 + 2ca_1^2 - ca_1^2) + a_1b_1(ca_1^2 - 2\|\vec{d_1}\|)z}{2\|\vec{d_1}\|^2(2\|\vec{d_1}\| - ca_1^2)}\\
       & = & \frac{\|\vec{d_1}\|B(2\|\vec{d_1}\| - ca_1^2) + a_1b_1(ca_1^2 - 2\|\vec{d_1}\|)z}{2\|\vec{d_1}\|^2(2\|\vec{d_1}\| - ca_1^2)}\\
       & = & \frac{\|\vec{d_1}\|B - a_1b_1z}{2\|\vec{d_1}\|^2}.
\end{eqnarray*}
We can now use the values of $a_2b_2$ and $a_2^2$ to find $b_2^2$.
\begin{eqnarray*}
b_2^2 & = & \frac{C - ca_2^2b_1^2 + 2ba_2b_1^2b_2}{\|\vec{d_1}\| + ab_1^2}\\
      & = & \frac{C - cb_1^2\left(\frac{2A\|\vec{d_1}\| - a_1^2z}{2\|\vec{d_1}\|^2} \right) + 2bb_1^2\left(\frac{\|\vec{d_1}\|B - a_1b_1z}{2\|\vec{d_1}\|^2}\right)}{\|\vec{d_1}\| + ab_1^2}\\
      & = & \frac{2\|\vec{d_1}\|^2C - cb_1^2(2A\|\vec{d_1}\| - a_1^2z) + 2bb_1^2(\|\vec{d_1}\|B - a_1b_1z)}{2\|\vec{d_1}\|^2(\|\vec{d_1}\| + ab_1^2)}\\
      & = & \frac{\|\vec{d_1}\|(2\|\vec{d_1}\|C - 2Acb_1^2 + 2bb_1^2B) + a_1b_1^2(ca_1 - 2bb_1)z}{2\|\vec{d_1}\|^2(\|\vec{d_1}\| + ab_1^2)}\\
      & = & \frac{\|\vec{d_1}\|(2Cab_1^2 - 4Cba_1b_1 + 2Cca_1^2 - 2Acb_1^2 + 2bb_1^2B) + a_1b_1^2(ca_1 - 2bb_1)z}{2\|\vec{d_1}\|^2(\|\vec{d_1}\| + ab_1^2)}\\
      & = & \frac{\|\vec{d_1}\|(4Cab_1^2 - 4Cba_1b_1 + 2Cca_1^2) + b_1^2(ca_1^2 - 2ba_1b_1 - 2\|\vec{d_1}\|)z}{2\|\vec{d_1}\|^2(\|\vec{d_1}\| + ab_1^2)}\\
      & = & \frac{2C\|\vec{d_1}\|(\|\vec{d_1}\| + ab_1^2) + b_1^2(\|\vec{d_1}\| + ab_1^2)z}{2\|\vec{d_1}\|^2(\|\vec{d_1}\| + ab_1^2)}\\
      & = & \frac{2C\|\vec{d_1}\| + b_1^2z}{2\|\vec{d_1}\|^2}.
\end{eqnarray*}
We finally take the ratio of $a_2^2$ and $a_2b_2$ to establish the appropriate roots to take when calculating values of $a_2$ and $b_2$.
\begin{eqnarray*}
\frac{a_2}{b_2} & = & \frac{a_2^2}{a_2b_2}\\
                & = & \frac{2A\|\vec{d_1}\| - a_1^2z}{B\|\vec{d_1}\| - a_1b_1z}.
\end{eqnarray*}
\end{proof}
This result motivates our original definition of the the diagonal vectors as we have expressions for the squares of $a_2$ and $b_2$.

\begin{theorem}\label{th:gram-diags}
Given a conic $\langle P\!:\!\vec{q}\!:\!r \rangle$ with diagonal vectors $\vec{d_1}$ and $\vec{d_2}$, the diagonals are $d_1$ and $d_2$ where
\begin{eqnarray}
c_1 & = & \frac{b_2D - a_2E}{2\|\vec{d_2}\|(a_1b_2 - a_2b_1)}\\
c_2 & = & \frac{a_1E - b_1D}{2\|\vec{d_1}\|(a_1b_2 - a_2b_1)}.
\end{eqnarray}
\end{theorem}
\begin{proof}Starting from equation (\ref{eq:gram-q}) we find
\begin{eqnarray*}
\vec{q} & = & 2\begin{pmatrix} a_1 & a_2 \\ b_1 & b_2 \end{pmatrix}\begin{pmatrix} \|\vec{d_2}\|c_1 \\ \|\vec{d_1}\|c_2  \end{pmatrix}\\
\begin{pmatrix} \|\vec{d_2}\|c_1 \\ \|\vec{d_1}\|c_2  \end{pmatrix} & = & \frac{1}{2}\begin{pmatrix} a_1 & a_2 \\ b_1 & b_2 \end{pmatrix}^{-1}\vec{q}\\
 & = & \frac{1}{2(a_1b_2 - a_2b_1)}\begin{pmatrix} -b_2 & -a_2 \\ -b_1 & a_1 \end{pmatrix} \begin{pmatrix} D \\ E\end{pmatrix}\\
 & = & \frac{1}{2(a_1b_2 - a_2b_1)}\begin{pmatrix} b_2D - a_2E \\ a_1E - b_1D\end{pmatrix}.
\end{eqnarray*}
\end{proof}

\begin{theorem}\label{th:gram-const}
Given a conic $\langle P\!:\!\vec{q}\!:\!r \rangle$ with diagonals $d_1$ and $d_2$, the grammola constant $K$ is given by
\begin{eqnarray}
K & = & \Delta_G\left(\frac{c_1^2}{\|\vec{d_1}\|} + \frac{c_2^2}{\|\vec{d_2}\|} - \frac{F}{\|\vec{d_1}\|\|\vec{d_2}\|}\right).
\end{eqnarray}
\end{theorem}
\begin{proof}
This result follows directly from equation (\ref{eq:gram-r}).
\end{proof}

\begin{corollary}
Given a conic $\langle P\!:\!\vec{q}\!:\!r \rangle$ and a single diagonal vector $\vec{d_1}$, we can calculate the diagonals $d_1$ and $d_2$ as well as the grammola constant $K$.
\end{corollary}
\begin{proof}
From Theorem \ref{th:gram-covec} we can find the co-diagonal vector $\vec{d_2}$. Theorem \ref{th:gram-diags} then lets us find the diagonals $d_1$ and $d_2$ which in turn can be used in Theorem \ref{th:gram-const} to find the constant $K$.
\end{proof}


\section{Quadrolas}

While grammolas are defined with respect to two lines, the diagonals, we define their counterpart, the quadrola, with respect to two points.

\begin{definition}Given two points, $X_1$ and $X_2$, and a constant $K$, a \emph{quadrola} is defined as the locus of points satisfying the equation $A(Q(X, X_1), Q(X, X_2), K) = 0$. 
\end{definition}
\begin{theorem}  Given two points, $X_0 = [x_0, y_0]$ and $X_1 = [x_1, y_1]$, and a constant $K$, the general form of their quadrola is $\langle P\!:\!\vec{q}\!:\!r \rangle$ where
\begin{eqnarray}
P & = & 4M(a\Delta x + b\Delta y, b\Delta x + c\Delta y) - 4KG\\
\vec{q} & = & 4(\vec{X_0}^2 - \vec{X_1}^2)G\Delta \vec{X} + 4KG(\vec{X_0} + \vec{X_1})\\
r & = & \left(K - \vec{X_0}^2 - \vec{X_1}^2\right)^2 - 4\vec{X_0}^2\vec{X_1}^2
\end{eqnarray}
and $\Delta \vec{X} = \vec{X_1} - \vec{X_0}$, $\Delta x = x_1 - x_0$ and $\Delta y = y_1 - y_0$.
\end{theorem}
\begin{proof} 
If we let $Q_1 = Q(X, X_1)$ and $Q_2 = Q(X, X_2)$, then from the definition and equation (\ref{eq:arch}) we have
\begin{eqnarray*}
0 & = & (Q_1 + Q_2 + K)^2 - 2(Q_1^2 + Q_2^2 + K^2)\nonumber\\
  & = & Q_1^2 + Q_2^2 + K^2 - 2(Q_1Q_2 + KQ_1 + KQ_2)\nonumber\\
  & = & (Q_1 - Q_2)^2 - 2K(Q_1 + Q_2) + K^2.\nonumber
\end{eqnarray*}
Expressing this in terms of $X$, and using metric dot products we have
\begin{eqnarray}
0  & = & \left(\left(\vec{X}^2 - 2(\vec{X}\cdot\vec{X_0}) + \vec{X_0}^2\right) - \left(\vec{X}^2 - 2(\vec{X}\cdot \vec{X_1}) + \vec{X_1}^2\right)\right)^2 \nonumber\\
  &   & - 2K\left(\left(\vec{X}^2 - 2(\vec{X}\cdot \vec{X_0}) + \vec{X_0}^2\right) + \left(\vec{X}^2 - 2(\vec{X}\cdot \vec{X_1}) + \vec{X_1}^2\right)\right) + K^2\nonumber\\
  & = & \left(2\vec{X}\cdot(\vec{X_1} - \vec{X_0}) + (\vec{X_0}^2 - \vec{X_1}^2)\right)^2 \nonumber\\
  &   & - 2K\left(2\vec{X}^2 - 2\vec{X}\cdot (\vec{X_0} + \vec{X_1}) + \vec{X_0}^2 + \vec{X_1}^2\right) + K^2\nonumber\\
  & = & \left(2\vec{X}\cdot(\vec{X_1} - \vec{X_0})\right)^2 + 4(\vec{X_0}^2 - \vec{X_1}^2)\vec{X}\cdot(\vec{X_1} - \vec{X_0}) + \left(\vec{X_0}^2 - \vec{X_1}^2\right)^2\nonumber\\
  &   & - 2K\left(2\vec{X}^2 - 2\vec{X}\cdot (\vec{X_0} + \vec{X_1}) + \vec{X_0}^2 + \vec{X_1}^2\right) + K^2\nonumber\\
  & = & 4\left(\vec{X}\cdot(\vec{X_1} - \vec{X_0})\right)^2 -4K\vec{X}^2\nonumber\\
  &   &  + 4(\vec{X_0}^2 - \vec{X_1}^2)\vec{X}\cdot(\vec{X_1} - \vec{X_0}) + 4K\vec{X}\cdot (\vec{X_0} + \vec{X_1})\nonumber\\
  &   &  - 2K(\vec{X_0}^2 + \vec{X_1}^2) + \left(\vec{X_0}^2 - \vec{X_1}^2\right)^2 + K^2\nonumber\\
  & = & 4\left(\left(\vec{X}\cdot(\vec{X_1} - \vec{X_0})\right)^2 -K\vec{X}^2\right)\nonumber\\
  &   &  + 4\left((\vec{X_0}^2 - \vec{X_1}^2)\vec{X}\cdot(\vec{X_1} - \vec{X_0}) + K\vec{X}\cdot (\vec{X_0} + \vec{X_1})\right)\nonumber\\
  &   &  + \left(K - \vec{X_0}^2 - \vec{X_1}^2\right)^2 - 4\vec{X_0}^2\vec{X_1}^2.\nonumber
\end{eqnarray}
\end{proof}

The inverse problem for a quadrola is significantly more difficult than for a grammola, since the equations we need to solve are quartic.
As such, the inverse problem is not addressed here, but would be a natural progression in the development of the theory of quadrolas in planar universal geometry.
